



\section{Sign Language -- {\small Stijn Janssens}}
According to the World Health Organisation, anno 2015, some 5\% of the world’s population is deaf. Different countries have their own sign language\idx{sign language}s, but these are often mutually unintelligible. Currently, no standard universal sign language\idx{sign language} exists. In our project, we did not focus our attention on creating a sign language\idx{sign language} to accompany Atlan as an IAL\index{international auxiliary language}, but we might suggest how others, who have more knowledgeable on this topic than us, might use Atlan to construct such a sign language\idx{sign language}. 

A database of signs from different sign language\idx{sign language}s might be used, such as ‘Spreadthesign’ by the European Sign Language Centre. Software such as Sign Language Processing (SLP) might be used to build data models to formalize and compare signs from different languages. This way, ‘universal’ signs might be generated by identifying overlap or similarity in signs between language, weighing each language by relatedness and amount of speakers. These could then be mapped onto Atlan’s lexicon, and from there the whole language might be essentially copied into this sign language\idx{sign language}.  

Having such a signing system might have the benefit of allowing deaf people from around  the word to communicate with one another. It might also make sign language\idx{sign language} more accessible to hearing people, who would only have to learn some 490 signs, provided they already speak Atlan. This would foster communication and mutual understanding between deaf and hearing people, as well as serving as an extra linguistic gadget for communication when two speakers can see each other, while unable to hear what the other is saying due to whatever circumstances. 

\section{Language Variation -- {\small Niek Elsinga}}

Imagine yourself in the following situation: You are standing in front of a machine which will take you back in time to the year 1223, 800 years in the past. Peace had just returned to England after a hard-fought civil war which resulted in the signing of the Magna Carta, which limited the power of the English kings. England in this time was a country that was still full of meadows, forests, and pristine nature, with a relatively small population of an estimated 4 million people, nearly 60 million fewer than today. You would be able to walk around , and enjoy a moment of tranquillity, peace, and quiet along crudely constructed cobbled walls which indicated roads that led to small villages, towns, and cities. While these towns might have been humble and quaint, they still bustled with life. People buzzing in tightly cramped avenues, with the smell\idx{smell}s of fresh crisp sourdough bread, savoury stews brewing above campfires, and the pungent aromas of leather tanners, and the fires of the bellows of blacksmiths must have all coalesced in the cacophony of the community. Shops, pubs, and artisanal boutiques which sell clothes and other food stuffs are able to be found  in dimly-lit alleyways. Market stalls with several kinds of fruits, vegetables, loaves of bread, meat, and perhaps even a mystic stall with unique herbs and spices from a faraway land are able to be found  on plaza on a Saturday morning. 

You walk up to a stall which sells different types of stew. While you are not entirely certain what it is in it, you are drawn to a certain type of stew which is simmering above a fire. Its aromas and smell\idx{smell}s are unlike you have seen thus far, and thus, you go ahead and order a portion of this seemingly taste\idx{taste}ful concoction. “I would like to order a portion of this stew, please,” you would say. The salesman looks you in the eye, astound ed and, perhaps, suspiciously. He replies: “Hwæne canst þú ġecwides?” You look dumbfound ed at the vendor. With every single word that you try to pronounce, it seems that his gaze turns more hostile. Eventually, you just point. “That one, please,” whilst pointing to a stew you did not even examine. You give him the money, which he luckily accepts, and hands you a bowl full of the other substance. This one smell\idx{smell}s significantly less refined than the other one, but you cannot be bothered to go back and voice your dissatisfaction. It was your fault either way, since you pointed erroneously. You sigh, and begrudgingly eat your stew which still turns out to be somewhat alright. 

 

What happened here? How come that you were not able to understand each other? In this case, there are two factors to this. First and foremost, language changes and branches out over time. This is normal, natural, and occurs organically. Because of language change, the Vulgar Latin of the Roman empire diverged and evolved into the modern Romance languages of Spanish, Portuguese, French, Italian, over the course of the last two millennia (Sala \& Posner, 1999). The same happened with Vedic Sanskrit, which is the now-extinct language from which a plethora of languages on the Indian subcontinent are derived from (Burde, 2004). 

The second factor is a variable that has happened in the English language specifically, which is a shift in the pronunciation of English vowels\idx{vowels}. The standardization of the English script occurred between the 15th and the 16th centuries (Denham \& Lobeck, 2009), while the pronunciation of English vowels\idx{vowels} shifted during this time. This shifting-event occurred between the 15th and 18th centuries, and influenced the pronunciation of vowels\idx{vowels} of every single English dialect (Labov, 1997). Where the vowels\idx{vowels} in the word “boot” are currently pronounced akin to the Dutch diphthong /oe/ in “koe”, or just the standard English [oo], in the 13th century it would have sound\idx{sound}ed more like the Dutch /oː/ as in “groot”, or the $\langle$aw$\rangle$\footnotemark in the modern British English word “yawn”. This Great Vowel Shift (GVS), as it is called, resulted in a different pronunciation compared to the graphemic notation for the entirety of the English language (Denham \& Lobeck, 2009). 

\footnotetext{These brackets are used for linguistic notations. $\langle\ldots\rangle$ is used for graphemic notation (i.e., the letters as they are written down); [...] is used for the actual realized phoneme (i.e., the sound\idx{sound} that is actually created); and /.../ is used for the intended phoneme.}

The GVS likely occurred because of multiple reasons, however, there is no academic consensus for one single solution (Silverman \& Silverman, 2012). Some theories include migrations towards the southeast of England from neighbouring regions following the population decline caused by the Black Death (Crystal, 2018). Another theory is the influx of French loanword\idx{loanword}s with differing pronunciation compared to the Anglo-Saxon pronunciation of Old and Early Middle English (Millward \& Hayes, 2011). Another theory is the complete opposite, which states that due to the wars with France in which England was entangled at that period in time, anti-French sentiment caused a shift in pronunciation to make English phonemes sound\idx{sound} less French (Nevalainen \& Traugott, 2012). It is more likely that the GVS occurred due a combination of these factors, rather than that a single one resulted in the entirety of the changes (Silverman \& Silverman, 2012). 

Nonetheless, it occurred, and English has not been the same since. It is not unlikely that events like the GVS will happen again since language is fluid per definition. Scholars agree on that language variation\idx{language variation} and change is both inevitable, unpreventable, and continuously happening (Lyons, 1968). In this chapter, I will elaborate on the specifics of language change, how it can occur, and how we have designed our language to be resistant to language variation\idx{language variation} and change to a certain degree. 


\subsection{Language variation\idx{language variation} and change: inevitable?}

Language variation\idx{language variation} refers to the different ways in which a language can vary based on factors such as geography, social groups, historical periods, and individual speakers. These variations can manifest in various forms, including pronunciation, vocabulary, grammar\idx{grammar}, and usage (O’Grady et al., 2001). Take regional dialects, for example. Different regions within a country or even different countries that share the same language may have distinct dialects. For instance, in Dutch, there are variations between the Dutch from the Netherlands and the Dutch from Flanders. These dialects can diverge in pronunciation (e.g., the pronunciation of the letter “g” and “r” in the Netherlands and Flanders (Verhoeven, 2005)), vocabulary (e.g., the use of the second-person pronoun “uw” in Flemish contraste\idx{taste}d with “jouw” in Dutch (Vandekerckhove, 2005)), and grammar\idx{grammar} (e.g., “moeten aan doen” in Flemish compared to “aan moeten doen” in Dutch (Haeseryn, 1990)). Another example is sociolects, which are variations based on social factors such as social class, education level, or occupation. There may be differences in vocabulary and speech patterns between a group of doctors and a group of construction workers, reflecting their professional background s and the jargon\idx{jargon} they use in their respective fields (Bybee, 2015; O’Grady et al., 2001) 

The main point in language variation\idx{language variation} is that variation is not the same as language change, however, language variation\idx{language variation} often does serve as a precursor to language change (Chambers et al., 2004). When a language exhibits variation among its speakers or regions, it provides the found ation for changes to occur and spread throughout a language community. Language change, in continuation of language variation\idx{language variation}, refers to the process by which a language undergoes modifications over time. There are multiple factors about language change, which can occur at every linguistic level: Phonology\idx{phonology} and phonetics, morphology\idx{morphology}, syntax\idx{syntax}, semantics\idx{semantics}, and pragmatics\idx{pragmatics} (Meecham \& Rees-Miller, 2001). Phonological change involves alterations in the sound\idx{sound}s of a language. Over time, sound\idx{sound}s can shift in pronunciation, merge with other sound\idx{sound}s, or split into distinct sound\idx{sound}s. This happens more frequently if multiple sound\idx{sound}s exist which sound\idx{sound} similar, such as the /$\theta$/ in $\langle$ thing$\rangle$ being replaced by the /f/. This happened to me personally, and occasionally I still make the error of pronouncing the $\langle$th$\rangle$ as an /f/ instead of the /$\theta$/. Lexi\idx{Lexi}cal change refers to changes in vocabulary. New words are constantly introduced into a language, while others become obsolete or change in meaning. For instance, the word “awful” originally meant “full of awe”, but has shifted to its current meaning of “bad” or "terrible" over time (“Awful, Adj. and Adv.: Oxford English Dictionary,” n.d.). Languages can also undergo changes in their grammatical structures. This includes modifications in verb conjugation, word order\idx{word order}, and the use of grammatical markers. Take for example the distinction with the indirect object “aan” in the Flemish “moeten aan doen” compared to the Dutch “aan moeten doen”, as stated earlier (Haeseryn, 1990). Semantic change occurs when the meaning of words or phrases evolves over time. Words can acquire new meanings, lose old meanings, or sustain shifts in connotation. An example is the word “gay”, which originally meant “happy” but has taken on the additional meaning of “homosexual” in modern usage (Hiskey, 2015). 

\subsection{Mechanics of variation and change}

Besides changes in language as part of coincidences of linguistic levels, change can also be instigated by social factors such as group identity and language contact. Social factors play a crucial role in shaping language variation\idx{language variation} and driving language change. Certain speech styles or dialects may be associated with social prestige, power, or higher social status. Speakers who want to align themselves with certain social classes may adopt features associated with these groups. Take for example the use of certain lexical items, jargon\idx{jargon}, or words on a semantic level. Using words associated with the desired group can give the illusion of being associated with said groups. As a result, language change can occur as features from prestigious or standard varieties are adopted and incorporated into the speech of a wider population (Labov, 1990). Besides class and income, speakers may also associate themselves with certain social groups, such as skaters, punks, emo’s, etc. Language is an important marker of social identity. Speakers may consciously or unconsciously modify their language use to identify with or differentiate themselves from particular social groups. Language change can occur as speakers apply features of the identity of the target group as a way to signal membership in a specific community or subculture. This happens oftentimes in groups of young adults, and as such, older individuals might not understand them (Coupland, 1985). Language change is also often observed between different generations. Younger speakers may introduce new linguistic innovations or modifications in their language use compared to older generations. Over time, as younger generations become the majority, their linguistic features may spread and become more widespread, leading to language change (Kerswill, 1996). However, within these older populations, language change can occur as well, through social networks. Perhaps some elderly individuals create a certain lect during their poker-games. Because speakers interact with others in their social networks, language change can be achieved through the innovation and diffusion of these linguistic innovations. Language change can occur when innovative linguistic features spread through social networks, especially if influential individuals or groups adopt and promote these features (Ke et al., 2008). More sinister causes of language changes can also occur. If a particular variation is stigmatized or associated with negative stereotypes, speakers may avoid using those features or modify their language use to conform to more prestigious or socially acceptable forms (Maass, 1999). The opposite can also occur, in that positive attitudes towards certain features can promote their adoption and spread, leading to language change. This strikes back at the aforementioned options. 

\subsection{Implications for Atlan in language development}

There are many reasons for both language variation\idx{language variation} and language change. Change and variation in language are inevitable (Aitchison, 1994). How does this fare against constructed languages then? Very few constructed languages have seen wide-spread implementations, or mass numbers of speakers. It seems that there is limited evidence for linguistic variation in Esperanto, the major constructed language (Sherwood, 1982). However, Sherwood (1982) solely found  variation in the pronunciation of phonemes, and there was still no mutual unintelligibility whatsoever. This is also likely due to the fact that Esperanto has seen no official adoption globally, and its use is mostly by aficionados (Piron, 1989). This causes the spoken language to be more or less the same as when it was invented, approximately 150 years ago. 

Treading the waters of future language variation\idx{language variation} can be a difficult subject, due to the fact that the future, simply put, cannot be predicted. Language variation\idx{language variation} and change is, of course, inevitable. However, we have taken steps in order to make Atlan more resistant to language change. This is mostly centred in the phonology\idx{phonology}: because there are cardinal groups for both vowels\idx{vowels} and consonants\idx{consonants} in which similar phonemes are both allophonic and grouped, variation will less likely occur on a phonemic level. The same is the case for morpho-syntax\idx{syntax} because prepositions, referents, demonstratives, etc. all have a fixed set and meaning, and syntax\idx{syntax}ial variation is allowed to a certain degree. Furthermore, because the lexicon is procedurally generated, but random by definition for other items, variation is more likely to occur due to the implementation of lexical items of the mother tongue of a speaker. This so- called L1-to-L2-transfer (Sparks et al., 2009), however, is a feature of Atlan. Because some lexical elements and words with complex meanings cannot be accurately translated due to cultural differences (House, 2010), speakers are encouraged to translate it literally, and perhaps elaborate on it to unknowing speakers. A good example of a word that has no direct literal translation\idx{translation} in English is the German word ‘Schadenfreude’. In Atlan, this word could be described as “joy (SUS \sus) + other (OF \of) + affect (SIN \Atlansin) + bad (PAK \pak) = SUS.OF.SIN.PAK” \sus \of \Atlansin \pak. The use of these lemmas implies that a negative occurrence caused another person, in this case the person speaking it, a certain degree of joy. By describing the source word in Atlan, it can be understood by a wider array of speakers who are not familiar with the term. Variation in this case then is more or less irrelevant unless the words themselves change meaning. However, because the lemmas are procedurally generated, variation can only occur if a pronunciation of a consonant or vowel is changed. And this, of course, is less likely due to the grouping of the consonants\idx{consonants} and vowels\idx{vowels} in their allophonic categories. Due to these considerations, we think that Atlan as a whole will likely experience a delayed progression of variation and change.

\subsection{Conclusion}

If anything is clear, it would be that language variation\idx{language variation} and change is inevitable, unpredictable in its course, and constantly occurring. Atlan, like every other language, will meet the same fate, and changes will occur, be it regionally, socio-economically, age or culture-related. Perhaps in the future, multiple different variations of Atlan will coexist, intelligible or unintelligible. Then, the decisions made for the mitigation of language variation\idx{language variation} and change will be in vain. However, is that not exciting? When language variation\idx{language variation} occurs, this means that it is alive and fluid. Being able to see a language flourish is, perhaps, a better outcome than rigid measures intended to keep the language intelligible for everyone.


