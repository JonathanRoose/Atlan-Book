


\section{The story of King Atlas}




\section{Linguistic relativity \\ {\small Max Geraerdts}}


\section{Need for an IAL \\ {\small Jonathan Roose}}

Historically the diversity of languages has been both a blessing and a curse. Not only is the verity of tongs a database of ways to understand the world and human expression, it also leads to barriers and in- and outgroups. This is why five of my co-students and I have taken up the ambitious task of creating an so-called International Auxiliary Language (IAL for short), a language that will allow its users to bridge language barriers and lead to mutual understanding between speakers with different mother tongues, a neutral ground on which all international communication can occur. The lingua franca’s of todays world that are used in intrenational relations, like French, English or Swahili give hierarchical stadings to the language of one particular group and/or state, these languages arebased on political power and historical conditions, they cannot be neutral, they are results of political interactions and thus are always politicisable. The aim of an IAL is to be a meeting ground of all people without it being dependent on power relations and historical animosities. This project has a lot in common with another IAL , Esperanto made by L.L Zamenhof as response to the violence between language groups in Europe. However where Esperanto succeded is also where it is limited. It did for a small part brought Europeans together however, only Europeans. Our goal with this project is to replicate this unifying international language but to the whole globe and not only limited to Europe.
The ambition we have with the language we call Atlan is to create a language that is based on nothing more that the human condition. Later in this book Stijn will explain more how we intend to do this however, for now I would like to introduce a term that might help to better understand what we hope to achieve with Atlan. A tertium comparationis is a wish of many translators is to have some way to compare the meaning of the original text with their translation. Of course the translation is meant to have the same meaning however, some meaning will always get lost, the comparison is to see whether the new meaning has not lost the essence of the original. The comparison is too see of the very quiddity of the original is captured in the translation. What translators want is a semiotic system that can show the quiddity , the essence of the message in a way that can be compared to natural languages. In Atlan we believe that we can create such a system by making a language based on the essential human experiences. A language that can get to the essence of a thing by basing it on the essential human ontological experience. This language will function as an IAL that is neutral and universal for it is a language based on the human condition that every human experiences.

\section{Eco's words \\ {\small Jonathan Roose}}

In this ambitious project we are indebted to the numerous projects that predate ours with the same or similar aims. Not only is there Zamenhof’s Esperanto many more thinkers have dealt with the quest for an IAL. To name all would be to numerous however we can mention a book that has introduced many of the language projects to us. Umberto Eco’s book The search for the Perfect Language has been a great source of knowledge in this project. Like Esperanto the book is mostly concerned with Europe. Notheless to finnish this introduction to Atlan we end with a passage from his book to summarise the project:
“Is it possible to reconcile the need for a common language and the need to defend linguistic heritages? Both of these needs reflect the same theoretical contradictions as well as the same practical possibilities. The limits of any international common language are the same as those of the natural languages on which these languages are modelled: all presuppose a principle of translatability. If a universal common language claims for itself the capacity to re-express a text written in any other language, it necessarily presumes that, despite the individual genius of any language, and despite the fact that each language constitutes its own rigid and unique way of seeing, organizing and interpreting the world, it is still always possible to translate from one language to another. However, if this is a prerequisite inherent to any universal language, it is at the same time a prerequisite inherent to any natural language. It is possible to translate from a natural language into a universal and artificial one for the same reasons that justify and guarantee the translation from a natural language into another. The intuition that the problem of translation itself presupposed a perfect language is already present in Walter Benjamin: since it is impossible to reproduce all the linguistic meaning of the source language into a target language, one is forced to place one’s faith in the convergence of all languages. In each language ‘taken as a whole, there is a self-identical thing that is meant, a thing which, nevertheless, is accessible to none of these languages taken individualy, but only to that totality of all of their intentions taken as reciprocal and complementary, a totality that we call Pure Language [reine Sprache]’ (Benjamin 1923)” (Eco 1995:345)
[explain why this quote and further passage]

